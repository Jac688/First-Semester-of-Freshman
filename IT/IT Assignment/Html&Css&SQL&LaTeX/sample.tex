\documentclass{article}
\usepackage{graphicx}
\usepackage{amsmath}
\title{Introduction to LaTeX}
\author{Bill Gates}
\date {November 2007}
\begin{document}
   \maketitle
\begin{abstract}
This article introduces the main features in LaTeX, including creating an
abstract, generating sections and subsections, composing mathematic
formulae, inserting ordered and unordered lists, inserting tables and
figures, and establish citations and references automatically.
\end{abstract}
\section{Main features}

Donald Knuth invented TeX in 1976 and wrote a book on TeX on 1984. Based on
TeX, Lamport Leslie created LaTeX and published a book on LaTeX in 1986. In
this section, we mainly introduce some fundamentals in LaTeX.
\subsection{Insert a table}
\begin{itemize}
\item Table with borders\\
(see Table \ref{TableHV})

\begin{table}[h]
\centering
 \begin{tabular}{|l|c|c|r|}
 \hline
 name & major & year & grade \\
 \hline
 John & Computer & 3 & A \\
 \hline
 Peter & Finance & 4 & A- \\
 \hline
 Tony & Physics & 2 & B \\
 \hline
 \end{tabular}
\caption{Table with borders}
\label{TableHV}
\end{table}
\item Table with no vertical lines\\
(see Table \ref{TableH})

\begin{table}[h]
 \centering
 \begin{tabular}{lccr}
 \hline
 name & major & year & grade \\
 \hline
 John & Computer & 3 & A \\
 \hline
 Peter & Finance & 4 & A- \\
 \hline
 Tony & Physics & 2 & B \\
 \hline
 \end{tabular}
\caption{Table with no vertical lines}
\label{TableH}
\end{table}

item Table with no horizonal lines\\
(see Table \ref{TableV})

\begin{table}[h]
 \centering
 \begin{tabular}{|l|c|c|r|}
 name & major & year & grade \\
 John & Computer & 3 & A \\
 Peter & Finance & 4 & A- \\
 Tony & Physics & 2 & B \\
 \end{tabular}
\caption{Table with no horizonal lines}
\label{TableV}
\end{table}\

\item Table with no borders\\
(see Table \ref{Table})

\begin{table}[h]
 \centering
 \begin{tabular}{lccr}
 name & major & year & grade \\
 John & Computer & 3 & A \\
 Peter & Finance & 4 & A- \\
 Tony & Physics & 2 & B \\
 \end{tabular}
\caption{Table with  no borders}
\label{Table}
\end{table}
\end{itemize}
\subsection{Insert a mathematic formula}
Following mathematic environments are most commonly used:
\begin{enumerate}
\item Power and indices examples.\\
$x^3$, $x_3$, $x^{3n}$, $x_{3n}$

\item Roots\\
$\sqrt[2]{x^2 + y^2}$

\item Sum and over brace\\
$\sum_{i=1}^{n}x_i$
\[ \sum_{i=1}^{n}x_i \]
\[\overbrace{x_1+x_2\cdots + x_n}^{n}\]

\item Fractions \\
$\frac{x-1}{x-2}$
\[ \frac{\frac{x-1}{y-1}+\frac{x+1}{y+1}}{\sum_{i=1}^{n}x_i}\]

\item Matrices \\
$X = \left(
\begin{array}{c c c}
1 & 0 & 1\\
1 & 1 & 1\\
0 & 0 & 0 \\
\end{array}
\right)$

\item Equations \\
\begin{equation}
\begin{split}
z &= (x+y)^2\\
  &= x^2+2xy+y^2\\
\end{split}
\end{equation}

\begin{displaymath}
f(x) = \left\{
\begin{array}{rr}
x^2 &\mbox{ if $x<0$} \\
x^3 &\mbox{ otherwise}
\end{array} \right.
\end{displaymath}

\end{enumerate}
\subsection{Insert a figure}
Figure \ref{Sample Picture} is a sample picture.
\begin{figure}[h]
\centering
\includegraphics[width=70mm,height=70mm]{forest.jpg}
\caption{Sample Picture}
\label{Sample Picture}
\end{figure}
\section{Conclusion}
\begin{thebibliography}{99}
\bibitem{Donald 1984} Donald Knuth, The TeX book, Massachusetts: Addison-Wesley, 1984
\bibitem{Lamport 1986} Lamport Leslie, LaTeX: A Document Preparation System. Addison-Wesley, 1986
\end{thebibliography}
\end{document}
